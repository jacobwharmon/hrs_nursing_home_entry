\documentclass[10pt]{beamer}

\usetheme[progressbar=frametitle]{metropolis}
\usepackage{appendixnumberbeamer}

\usepackage{booktabs}
\usepackage[scale=2]{ccicons}

\usepackage{pgfplots}
\usepgfplotslibrary{dateplot}

\usepackage{graphicx}
\definecolor{firebrick}{rgb}{0.7, 0.13, 0.13}
\definecolor{dodgerblue}{rgb}{0.12, 0.56, 1.0}

\usepackage{tikz}
\usetikzlibrary{positioning}

\usepackage{xspace}
\newcommand{\themename}{\textbf{\textsc{metropolis}}\xspace}

\title{Health Inequality and Economic Disparities by Race, Ethnicity, and Gender}
\date{\today}
\author{Nicolò Russo, Rory McGee, Mariacristina De Nardi, Margherita Borella, and Ross Abram}
\institute{Presented by Jake Harmon}

\begin{document}

\maketitle

\begin{frame}{Introduction: Objective}
\large{
    \begin{itemize}
        \item How unequally distributed is health by race, ethnicity, and gender? \\
    \end{itemize}
}
\end{frame}

\begin{frame}{Introduction: Objective}
\large{
    \begin{itemize}
        \item How unequally distributed is health by race, ethnicity, and gender? \\
        \item How can these disparities explain differences in key economic outcomes such as:
        \begin{itemize}
            \item Disability
            \item Length of working life
            \item Nursing home entry
            \item Duration of life spent in poor health
            \item Lifespan
        \end{itemize}
    \end{itemize}
}
\end{frame}

\begin{frame}{Data}
    Health and Retirement Study (HRS) 
    \begin{itemize}
        \item Longitudinal survey of a representative sample of Americans over age 50 \\
        \item Participants interviewed every 2 years \\
        \item Participants provide data on health care, housing, assets, pensions, employment and disability
    \end{itemize}
\end{frame}

\begin{frame}{Data}
    RAND HRS Longitudinal File combines individual survey data into single 1992-2018 dataset
    \begin{itemize}
        \item Data used is 1996-2018 (key variables not tracked until 1996) \\
        \item Looking only at participants under age 100 who identify as non-Hispanic White, non-Hispanic Black, or Hispanic \\
        \item Sample consists of 216,166 individual-year observations
    \end{itemize}
\end{frame}

\begin{frame}{Methodology}
    \begin{enumerate}
        \item Construct measure of frailty (fraction of one's possible health deficits) \\
        \item Correct for under-reporting of medical diagnoses for Black and Hispanic individuals \\
        \item Estimate differences in key economic outcomes
    \end{enumerate}
\end{frame}

\begin{frame}{Methodology I: Frailty}
    \hspace*{0.2cm}
    \includegraphics[scale=0.5]{paper_tables/table_1_health_deficits.png}
\end{frame}

\begin{frame}{Methodology II: Correct for Under-Reporting}
    \hspace*{-0.75cm}
    \includegraphics[scale=0.42]{paper_tables/figure_A1_imputation.png}
\end{frame}

\begin{frame}{Results}
    \hspace*{-0.5cm}
    \includegraphics[scale=0.5]{paper_tables/figure_1a_health_deficit_prevalence_women.png}
\end{frame}

\begin{frame}{Results}
    \hspace*{-0.5cm}
    \includegraphics[scale=0.5]{paper_tables/figure_1b_health_deficit_prevalence_men.png}
\end{frame}

\begin{frame}{Results}
    \hspace*{-0.5cm}
    \includegraphics[scale=0.475]{paper_tables/figure_2_difference_in_health_deficit_prevalence.png}
\end{frame}

\begin{frame}{Results}
    \hspace*{-0.25cm}
    \includegraphics[scale=0.5]{paper_tables/figure_3a_potential_frailty_women.png}
\end{frame}

\begin{frame}{Results}
    \hspace*{-0.25cm}
    \includegraphics[scale=0.525]{paper_tables/figure_3b_potential_frailty_men.png}
\end{frame}

\begin{frame}{Results}
    \hspace*{0.85cm}
    \includegraphics[scale=0.35]{paper_tables/table_2_pseudo_rquared.png}
\end{frame}

\begin{frame}{Results}
    \hspace*{0cm}
    \includegraphics[scale=0.45]{paper_tables/figure_5_frailty_percentiles.png}
\end{frame}

\begin{frame}{Results}
    \hspace*{0cm}
    \includegraphics[scale=0.425]{paper_tables/figure_6_frailty_comparisons.png}
\end{frame}

\begin{frame}{Results}
    \hspace*{0cm}
    \includegraphics[scale=0.5]{paper_tables/figure_8_future_outcomes_bad_health.png}
\end{frame}

\begin{frame}{Results}
    \hspace*{0cm}
    \includegraphics[scale=0.5]{paper_tables/figure_8_future_outcomes_disability_benefits.png}
\end{frame}

\begin{frame}{Results}
    \hspace*{0cm}
    \includegraphics[scale=0.5]{paper_tables/figure_8_future_outcomes_life_expectancy.png}
\end{frame}

\begin{frame}{Results}
    \hspace*{0cm}
    \includegraphics[scale=0.5]{paper_tables/figure_8_future_outcomes_nursing_home.png}
\end{frame}

\begin{frame}{Results}
    \begin{enumerate}
        \item Evidence of substantial health inequality \\
        \item White men and women also have much lower frailty (i.e., better health), on average, than Hispanic and Black ones
    \end{enumerate}
\end{frame}

\begin{frame}{Limitations: Nursing Home}
    "Factors like informal care from extended family may
have a greater impact. For instance, Almeida, Molnar, Kawachi, and Subramanian (2009)
shows that Hispanic Americans have large family networks and high levels of social support,
which may explain why they spend less time in nursing homes than their White and Black
counterparts."
\end{frame}

\begin{frame}{Extension: Objective}
    \begin{enumerate}
        \item Incorporate additional measures of family structure \\
        \item Incorporate additional measures of income and assets \\
        \item Improve existing predictions of nursing home entry with additional measures \\
        \item Logistic regression v. machine learning
    \end{enumerate}
\end{frame}

\begin{frame}{Extension: Model}
    \[Pr(nursinghome_{i,t+1}=1)=NH(h_{it},X_{i,t})\]
    Where
    \begin{itemize}
        \item h is estimated health transition probabilities \\
        \item X is the set of basic controls, self reported health status, and frailty
    \end{itemize}
\end{frame}

\begin{frame}{Extension: Model}
    \[Pr(nursinghome_{i,t+1}=1)=NH(h_{it},X_{i,t}, \alert{Z_{i,t}})\]
    Where
    \begin{itemize}
        \item h is estimated health transition probabilities \\
        \item X is the set of basic controls, self reported health status, and frailty \\
        \alert{\item Z is the set of variables measuring family structure and income and assets}
    \end{itemize}
\end{frame}

\begin{frame}{Extension: Data}
    \hspace*{-0.5cm}
    \includegraphics[scale=0.475]{tables/table_1_summary_statistics.png}
\end{frame}

\begin{frame}{Extension: Methodology}
    \begin{itemize}
        \item Exclude individuals that lived in nursing home last wave or before \\
        \item Oversampling the minority group of nursing home entrants
    \end{itemize}
\end{frame}

\begin{frame}{Extension: Methodology}
    \hspace*{-0.85cm}
    \includegraphics[scale=0.325]{tables/figure_1_nursing_home_entry.png}
\end{frame}

\begin{frame}{Extension: Results}
    \hspace*{-0.5cm}
    \includegraphics[scale=0.475]{tables/table_2_pseudo_r_squared.png}
\end{frame}

\begin{frame}{Extension: Machine Learning}
    \begin{itemize}
        \item L1/L2 Regularization (Elastic Net)
        \item Random Forest
        \item Gradient Boosting
    \end{itemize}
\end{frame}

\begin{frame}{Extension: Machine Learning Results}
    \hspace*{-0.5cm}
    \includegraphics[scale=0.475]{tables/table_3_model_comparisons_pseudo_r_squared.png}
\end{frame}

\begin{frame}{Extension: Machine Learning Results}
    \hspace*{-0.5cm}
    \includegraphics[scale=0.475]{tables/table_3_model_comparisons_pseudo_r_squared.png}
    \newline
    \newline
    "McFadden (1977) argues that values between 0.2 and 0.4 denote an “excellent fit” of the full model."
\end{frame}


\begin{frame}{Extension: Machine Learning Results}
    \begin{align*}
        F_1 \text{ score} & = 2 \cdot \frac{\text{Precision} \cdot \text{Recall}}{\text{Precision} + \text{Recall}} \\
        & \\
        & \\
        \text{Where:         } & \\
        \text{Precision} & = \frac{\text{True Positives}}{\text{True Positives} + \text{False Positives}} \\
        \text{Recall} & = \frac{\text{True Positives}}{\text{True Positives} + \text{False Negatives}}
    \end{align*}
\end{frame}

\begin{frame}{Extension: Machine Learning Results}
    \hspace*{-0.55cm}
    \includegraphics[scale=0.475]{tables/table_4_model_comparisons_f1.png}
\end{frame}

\begin{frame}{Extension: Summary}
    \begin{enumerate}
        \item Incorporate additional measures of family structure \\
        \item Incorporate additional measures of income and assets \\
        \item Improve existing predictions of nursing home entry with additional measures \\
        \item Logistic regression v. machine learning
    \end{enumerate}
\end{frame}



\end{document}
